%%%%%%%%%%%%%%%%%%%%%%%%%%%%%%%%%%%%%%%%%
% The Legrand Orange Book
% Configurations File
% Version 2.1 (26/09/2018)
%
% Original author:
% Mathias Legrand (legrand.mathias@gmail.com) with modifications by:
% Vel (vel@latextemplates.com)
%
% This file was downloaded from:
% http://www.LaTeXTemplates.com
%
% License:
% CC BY-NC-SA 3.0 (http://creativecommons.org/licenses/by-nc-sa/3.0/)
%
%%%%%%%%%%%%%%%%%%%%%%%%%%%%%%%%%%%%%%%%%

%----------------------------------------------------------------------------------------
%	VARIOUS REQUIRED PACKAGES AND CONFIGURATIONS
%----------------------------------------------------------------------------------------

\usepackage{graphicx} % Required for including pictures
	\graphicspath{{Pictures/}} % Specifies the directory where pictures are stored

\usepackage{lipsum} % Inserts dummy text

\usepackage{verbatim} % Allows multiline comments

\usepackage{longtable}

\usepackage{tikz} % Required for drawing custom shapes
	\usetikzlibrary{patterns,decorations.pathreplacing} % for integral diagram
	\usetikzlibrary{decorations.markings} % for integral diagram

\usepackage{tikz-cd} % Required for drawing commutative diagrams

\usepackage{tikz-3dplot} % Required for drawing 3d plots


\usepackage{nicematrix} % Allows highlighting rows and columns of matrices
\usepackage[customcolors]{hf-tikz}

	
\usepackage{mathtools} % Required for DeclarePairedDelimiter

\usepackage[english]{babel} % English language/hyphenation

\usepackage[24hr,iso]{datetime} % Required for time and \today formats

\usepackage{enumitem} % Customize lists
	\setlist{nolistsep} % Reduce spacing between bullet points and numbered lists

\usepackage{booktabs} % Required for nicer horizontal rules in tables

\usepackage{pgfplots} % For plotting functions
	\usepgfplotslibrary{colormaps} % for plotting vector fields
	\usepgfplotslibrary{fillbetween} % for filling regions
	% Define a common style for plotting functions named 'myCommonStyle'
	\pgfplotsset{myCommonStyle/.append style={
	anchor=origin, 
	disabledatascaling, 
	grid=both, 
	grid style={line width=.1pt, draw=gray!10}, 
	axis lines=middle, 
	minor tick num=0,
	enlargelimits={abs=0.5}, 
	axis line style={latex-latex}, 
	ticklabel style={font=\tiny,fill=white}, 
	xlabel style={at={(ticklabel* cs:1)},anchor=west},
	ylabel style={at={(ticklabel* cs:1)},anchor=south}}
	}
	% Define a common style for plotting vector fields named 'MyVfieldStyle'
	\pgfplotsset{MyVfieldStyle/.append style={
		anchor=origin,
		disabledatascaling,
		grid=both,
		grid style={line width=.1pt, draw=gray!10},
		axis lines=left,
		minor tick num=0,
		%enlargelimits={abs=0.5},
		axis line style={latex-latex},
		ticklabel style={font=\tiny,fill=white},
		xlabel style={at={(ticklabel* cs:1)},anchor=west},
		ylabel style={at={(ticklabel* cs:1)},anchor=west}, 
		view={0}{90},
		% colormap/jet,
		% colorbar,
		% colorbar style = {ylabel = {Vector Magnitude}},
		zmin = -1, zmax = 0}
	}

\usepackage{xcolor} % Required for specifying colors by name
	\definecolor{DarkBlue}{HTML}{0077BB} 
	\definecolor{LightBlue}{HTML}{007FA3}
	\definecolor{Yellow}{HTML}{A37500}
	\definecolor{Red}{HTML}{CC3311} 
	\definecolor{Purple}{HTML}{6500A3}
	% Coloring for Example 6.1.8
	\definecolor{MySubdividesYellow}{HTML}{FFFF99}
	\definecolor{MySubdividesBlue}{HTML}{99CCFF}
	\definecolor{MySubdividesRed}{HTML}{FFCCCC}
	\definecolor{MySubdividesBlueYellow}{HTML}{D1E8C7}
	\definecolor{MySubdividesBlueRed}{HTML}{d6cce0}
	% Coloring for an example in 1.2 Scalar fields
	\definecolor{MyContourBlue}{HTML}{2626AD}
	\definecolor{MyContourYellowBlue}{HTML}{C4C424}
	\definecolor{MyContourRedYellow}{HTML}{FF7000}

	
	% These colours are good for colourblind people
	% https://personal.sron.nl/~pault/
	\definecolor{tolBlue}{HTML}{0077BB} 
	\definecolor{tolCyan}{HTML}{33BBEE}
	\definecolor{tolTeal}{HTML}{009988} 
	\definecolor{tolOrange}{HTML}{EE7733} 
	\definecolor{tolRed}{HTML}{CC3311} 
	\definecolor{tolMagenta}{HTML}{EE3377} 
	\definecolor{tolGrey}{HTML}{BBBBBB} 
	\definecolor{tolPurple}{HTML}{AA3377}
	\definecolor{tolSand}{HTML}{DDCC77}

\usepackage{colortbl}

\tikzset{style blue/.style={
    set fill color=tolBlue!30,
    set border color = tolBlue!30
  },
  style orange/.style={
    set fill color=tolOrange!40,
    set border color = tolOrange!40
  },
  style red/.style={
    ,set fill color=tolRed!40,
    set border color = tolRed!40
  },
  style teal/.style={
    set fill color=tolTeal!40,
    set border color = tolTeal!40
  },
  style purple/.style={
    set fill color=tolPurple!40,
    set border color = tolPurple!40
  },
  hor/.style={
    above left offset={-0.1,0.31},
    below right offset={0.1,-0.125},
    #1
  },
  ver/.style={
    above left offset={-0.1,0.3},
    below right offset={0.1,-0.15},
    #1
  }
}
%----------------------------------------------------------------------------------------
%	MARGINS
%----------------------------------------------------------------------------------------

\usepackage{geometry} % Required for adjusting page dimensions and margins

\geometry{
	paper=a4paper, % Paper size, change to letterpaper for US letter size
	top=3cm, % Top margin
	bottom=3cm, % Bottom margin
	left=3cm, % Left margin
	right=3cm, % Right margin
	headheight=14pt, % Header height
	footskip=1.4cm, % Space from the bottom margin to the baseline of the footer
	headsep=10pt, % Space from the top margin to the baseline of the header
	%showframe, % Uncomment to show how the type block is set on the page
}

%----------------------------------------------------------------------------------------
%	FONTS
%----------------------------------------------------------------------------------------

\usepackage{avant} % Use the Avantgarde font for headings and default text
\usepackage[charter,cal=cmcal]{mathdesign} % Use the Charter font for math
\usepackage{microtype} % Slightly tweak font spacing for aesthetics
\usepackage[utf8]{inputenc} % Required for including letters with accents
\usepackage[T1]{fontenc} % Use 8-bit encoding that has 256 glyphs


%----------------------------------------------------------------------------------------
%	BIBLIOGRAPHY AND INDEX
%----------------------------------------------------------------------------------------

% Bibliography
\usepackage[style=numeric,citestyle=numeric,sorting=nyt,sortcites=true,autopunct=true,babel=hyphen,hyperref=true,abbreviate=true,backref=true,backend=biber]{biblatex}
\addbibresource{bibliography.bib} % BibTeX bibliography file
\defbibheading{bibempty}{}

% Index
\usepackage{calc} % For simpler calculation - used for spacing the index letter headings correctly
\usepackage{makeidx} % Required to make an index
\makeindex % Tells LaTeX to create the files required for indexing



%----------------------------------------------------------------------------------------
%	MAIN TABLE OF CONTENTS
%----------------------------------------------------------------------------------------

\usepackage{titletoc} % Required for manipulating the table of contents

\contentsmargin{0cm} % Removes the default margin

% Part text styling (this is mostly taken care of in the PART HEADINGS section of this file)
\titlecontents{part}
	[0cm] % Left indentation
	{\addvspace{20pt}\bfseries} % Spacing and font options for parts
	{}
	{\color{white}}
	{}

% Chapter text styling
\titlecontents{chapter}
	[1.25cm] % Left indentation
	{\addvspace{12pt}\large\sffamily\bfseries} % Spacing and font options for chapters
	{\color{LightBlue!60}\contentslabel[\Large\thecontentslabel]{1.25cm}\color{LightBlue}} % Formatting of numbered sections of this type
	{\color{LightBlue}} % Formatting of numberless sections of this type
	{\color{LightBlue!60}\normalsize\;\titlerule*[.5pc]{.}\;\thecontentspage} % Formatting of the filler to the right of the heading and the page number

% Section text styling
\titlecontents{section}
	[1.25cm] % Left indentation
	{\addvspace{3pt}\sffamily\bfseries\color{DarkBlue}} % Spacing and font options for sections
	{\contentslabel[\thecontentslabel]{1.25cm}\color{DarkBlue}} % Formatting of numbered sections of this type
	{\color{DarkBlue}} % Formatting of numberless sections of this type
	{\hfill\color{black}\thecontentspage} % Formatting of the filler to the right of the heading and the page number

% Subsection text styling
\titlecontents{subsection}
	[1.25cm] % Left indentation
	{\addvspace{1pt}\sffamily\small} % Spacing and font options for subsections
	{\contentslabel[\thecontentslabel]{1.25cm}} % Formatting of numbered sections of this type
	{} % Formatting of numberless sections of this type
	{\ \titlerule*[.5pc]{.}\;\thecontentspage} % Formatting of the filler to the right of the heading and the page number

% Figure text styling
\titlecontents{figure}
	[1.25cm] % Left indentation
	{\addvspace{1pt}\sffamily\small} % Spacing and font options for figures
	{\thecontentslabel\hspace*{1em}} % Formatting of numbered sections of this type
	{} % Formatting of numberless sections of this type
	{\ \titlerule*[.5pc]{.}\;\thecontentspage} % Formatting of the filler to the right of the heading and the page number

% Table text styling
\titlecontents{table}
	[1.25cm] % Left indentation
	{\addvspace{1pt}\sffamily\small} % Spacing and font options for tables
	{\thecontentslabel\hspace*{1em}} % Formatting of numbered sections of this type
	{} % Formatting of numberless sections of this type
	{\ \titlerule*[.5pc]{.}\;\thecontentspage} % Formatting of the filler to the right of the heading and the page number

%----------------------------------------------------------------------------------------
%	MINI TABLE OF CONTENTS IN PART HEADS
%----------------------------------------------------------------------------------------

% Chapter text styling
\titlecontents{lchapter}
	[0em] % Left indentation
	{\addvspace{15pt}\large\sffamily\bfseries} % Spacing and font options for chapters
	{\color{LightBlue}\contentslabel[\Large\thecontentslabel]{1.25cm}\color{LightBlue}} % Chapter number
	{}
	{\color{LightBlue}\normalsize\sffamily\bfseries\;\titlerule*[.5pc]{.}\;\thecontentspage} % Page number

% Section text styling
\titlecontents{lsection}
	[0em] % Left indentation
	{\sffamily\small} % Spacing and font options for sections
	{\contentslabel[\thecontentslabel]{1.25cm}} % Section number
	{}
	{}

% Subsection text styling (note these aren't shown by default, display them by searchings this file for tocdepth and reading the commented text)
\titlecontents{lsubsection}
	[.5em] % Left indentation
	{\sffamily\footnotesize} % Spacing and font options for subsections
	{\contentslabel[\thecontentslabel]{1.25cm}}
	{}
	{}

%----------------------------------------------------------------------------------------
%	HEADERS AND FOOTERS
%----------------------------------------------------------------------------------------

\usepackage{fancyhdr} % Required for header and footer configuration

\pagestyle{fancy} % Enable the custom headers and footers

\renewcommand{\chaptermark}[1]{\markboth{\sffamily\normalsize\bfseries\chaptername\ \thechapter.\ #1}{}} % Styling for the current chapter in the header
\renewcommand{\sectionmark}[1]{\markright{\sffamily\normalsize\thesection\hspace{5pt}#1}{}} % Styling for the current section in the header

\fancyhf{} % Clear default headers and footers
\fancyhead[LE,RO]{\sffamily\normalsize\thepage} % Styling for the page number in the header
\fancyhead[LO]{\rightmark} % Print the nearest section name on the left side of odd pages
\fancyhead[RE]{\leftmark} % Print the current chapter name on the right side of even pages
%\fancyfoot[C]{\thepage} % Uncomment to include a footer
%\fancyfoot[R]{\copyright\,Asif Zaman, 2021}
%\fancyfoot[R]{\footnotesize \copyright\, \theauthor, \the\year \ \makebox(30,5){\includegraphics[height=1.2em]{copyright.png}}}

\renewcommand{\headrulewidth}{0.5pt} % Thickness of the rule under the header

\fancypagestyle{plain}{% Style for when a plain pagestyle is specified
	\fancyhead{}\renewcommand{\headrulewidth}{0pt}%
}

% Removes the header from odd empty pages at the end of chapters
\makeatletter
\renewcommand{\cleardoublepage}{
\clearpage\ifodd\c@page\else
\hbox{}
\vspace*{\fill}
\thispagestyle{empty}
\newpage
\fi}

%----------------------------------------------------------------------------------------
%	LINKS AND REFERENCES
%----------------------------------------------------------------------------------------

% Required to load after 'titletoc' package and before 'cleveref' package
\usepackage{hyperref} 
\hypersetup{hidelinks,backref=true,pagebackref=true,hyperindex=true,colorlinks=false,breaklinks=true,urlcolor=Red,linkcolor=LightBlue,bookmarks=true,bookmarksopen=false}
\newcommand{\xref}[2]{\href{#1}{\texttt{\color{Red}#2}}}


\usepackage{bookmark}
\bookmarksetup{
open,
numbered,
addtohook={%
\ifnum\bookmarkget{level}=0 % chapter
\bookmarksetup{bold}%
\fi
\ifnum\bookmarkget{level}=-1 % part
\bookmarksetup{color=LightBlue,bold}%
\fi
}
}



%----------------------------------------------------------------------------------------
%	THEOREM STYLES
%----------------------------------------------------------------------------------------

\usepackage{amsmath,amsfonts,amsthm} % For math equations, theorems, symbols, etc; exclude amssymb for package conflict with mathdesign

% Required to load before \newtheorem and after 'amsthm' package
\usepackage[noabbrev,capitalise,nameinlink]{cleveref} % Auto-reference theorems by name
	\crefname{theoremT}{Theorem}{Theorems}
	\crefname{corollaryT}{Corollary}{Corollaries}
	\crefname{exerciseT}{Exercise}{Exercises}
	\crefname{lemmaT}{Lemma}{Lemmas}
	\crefname{exampleT}{Example}{Examples}
	\crefname{definitionT}{Definition}{Definitions}
	\crefname{remarkT}{Remark}{Remarks}


\numberwithin{equation}{section} % Number equations within section
\newcommand{\intoo}[2]{\mathopen{]}#1\,;#2\mathclose{[}}
\newcommand{\ud}{\mathop{\mathrm{{}d}}\mathopen{}}
\newcommand{\intff}[2]{\mathopen{[}#1\,;#2\mathclose{]}}
\renewcommand{\qedsymbol}{$\blacksquare$}
\newtheorem{notation}{Notation}[chapter]

% Boxed/framed environments
\newtheoremstyle{bluenumbox}% Theorem style name
{0pt}% Space above
{0pt}% Space below
{\normalfont}% Body font
{}% Indent amount
{\small\bf\sffamily\color{LightBlue}}% Theorem head font
{\;}% Punctuation after theorem head
{0.25em}% Space after theorem head
{\small\sffamily\color{LightBlue}\thmname{#1}\nobreakspace\thmnumber{\@ifnotempty{#1}{}\@upn{#2}}% Theorem text (e.g. Theorem 2.1)
\thmnote{\nobreakspace\the\thm@notefont\sffamily\color{LightBlue}(#3)}} % Optional theorem note

\newtheoremstyle{bluenumex}% Theorem style name
{0pt}% Space above
{0pt}% Space below
{\normalfont}% Body font
{}% Indent amount
{\footnotesize\bf\sffamily\color{tolMagenta}}% Theorem head font
{\;}% Punctuation after theorem head
{0.25em}% Space after theorem head
{\footnotesize\sffamily\color{tolMagenta}\thmname{#1}\nobreakspace\thmnumber{\@ifnotempty{#1}{}\@upn{#2}}% Theorem text (e.g. Theorem 2.1)
\thmnote{\nobreakspace\the\thm@notefont\sffamily\color{tolMagenta}(#3)}} % Optional theorem note

\newtheoremstyle{darkbluenumbox}% Theorem style name
{0pt}% Space above
{0pt}% Space below
{\normalfont}% Body font
{}% Indent amount
{\small\bf\sffamily\color{DarkBlue}}% Theorem head font
{\;}% Punctuation after theorem head
{0.25em}% Space after theorem head
{\small\sffamily\color{DarkBlue}\thmname{#1}\nobreakspace\thmnumber{\@ifnotempty{#1}{}\@upn{#2}}% Theorem text (e.g. Theorem 2.1)
\thmnote{\nobreakspace\the\thm@notefont\sffamily\color{DarkBlue}(#3)}} % Optional theorem note

\newtheoremstyle{darkbluenumex}% Theorem style name
{0pt}% Space above
{0pt}% Space below
{\normalfont}% Body font
{}% Indent amount
{\footnotesize\bf\sffamily\color{tolMagenta}}% Theorem head font
{\;}% Punctuation after theorem head
{0.25em}% Space after theorem head
{\footnotesize\sffamily\color{tolMagenta}\thmname{#1}\nobreakspace\thmnumber{\@ifnotempty{#1}{}\@upn{#2}}% Theorem text (e.g. Theorem 2.1)
\thmnote{\nobreakspace\the\thm@notefont\sffamily\color{tolMagenta}(#3)}} % Optional theorem note

\newtheoremstyle{yellownumbox}% Theorem style name
{0pt}% Space above
{0pt}% Space below
{\normalfont}% Body font
{}% Indent amount
{\small\bf\sffamily\color{tolRed}}% Theorem head font
{\;}% Punctuation after theorem head
{0.25em}% Space after theorem head
{\small\sffamily\color{tolRed}\thmname{#1}\nobreakspace\thmnumber{\@ifnotempty{#1}{}\@upn{#2}}% Theorem text (e.g. Theorem 2.1)
\thmnote{\nobreakspace\the\thm@notefont\sffamily\color{tolRed}(#3)}} % Optional theorem note

\newtheoremstyle{yellownumex}% Theorem style name
{0pt}% Space above
{0pt}% Space below
{\normalfont}% Body font
{}% Indent amount
{\footnotesize\bf\sffamily\color{tolOrange}}% Theorem head font
{\;}% Punctuation after theorem head
{0.25em}% Space after theorem head
{\footnotesize\sffamily\color{tolOrange}\thmname{#1}\nobreakspace\thmnumber{\@ifnotempty{#1}{}\@upn{#2}}% Theorem text (e.g. Theorem 2.1)
\thmnote{\nobreakspace\the\thm@notefont\sffamily\color{tolOrange}(#3)}} % Optional theorem note

\newtheoremstyle{rednumbox}% Theorem style name
{0pt}% Space above
{0pt}% Space below
{\normalfont}% Body font
{}% Indent amount
{\small\bf\sffamily\color{tolBlue}}% Theorem head font
{\;}% Punctuation after theorem head
{0.25em}% Space after theorem head
{\small\sffamily\color{tolBlue}\thmname{#1}\nobreakspace\thmnumber{\@ifnotempty{#1}{}\@upn{#2}}% Theorem text (e.g. Theorem 2.1)
\thmnote{\nobreakspace\the\thm@notefont\sffamily\color{tolBlue}(#3)}} % Optional theorem note

\newtheoremstyle{rednumex}% Theorem style name
{0pt}% Space above
{0pt}% Space below
{\normalfont}% Body font
{}% Indent amount
{\footnotesize\bf\sffamily\color{Red}}% Theorem head font
{\;}% Punctuation after theorem head
{0.25em}% Space after theorem head
{\footnotesize\sffamily\color{Red}\thmname{#1}\nobreakspace\thmnumber{\@ifnotempty{#1}{}\@upn{#2}}% Theorem text (e.g. Theorem 2.1)
\thmnote{\nobreakspace\the\thm@notefont\sffamily\color{Red}(#3)}} % Optional theorem note

\newtheoremstyle{blacknumbox} % Theorem style name
{0pt}% Space above
{0pt}% Space below
{\normalfont}% Body font
{}% Indent amount
{\small\bf\sffamily\color{LightBlue}}% Theorem head font
{\;}% Punctuation after theorem head
{0.25em}% Space after theorem head
{\small\sffamily\thmname{#1}\nobreakspace\thmnumber{\@ifnotempty{#1}{}\@upn{#2}}% Theorem text (e.g. Theorem 2.1)
\thmnote{\nobreakspace\the\thm@notefont\sffamily(#3)}} % Optional theorem note

\newtheoremstyle{blacknumex}% Theorem style name
{0pt}% Space above
{0pt}% Space below
{\normalfont}% Body font
{} % Indent amount
{\footnotesize\bf\sffamily}% Theorem head font
{\;}% Punctuation after theorem head
{0.25em}% Space after theorem head
{\footnotesize\sffamily{\tiny\ensuremath{\blacksquare}}\nobreakspace\thmname{#1}\nobreakspace\thmnumber{\@ifnotempty{#1}{}\@upn{#2}}% Theorem text (e.g. Theorem 2.1)
\thmnote{\nobreakspace\the\thm@notefont\sffamily(#3)}} % Optional theorem note

% Defines the theorem text style for each type of theorem to one of the three styles above

% blue styles
\theoremstyle{bluenumbox}
\newtheorem{theoremT}{Theorem}[section]
\newtheorem{corollaryT}[theoremT]{Corollary}
\theoremstyle{bluenumex}
\newtheorem{exerciseT}[theoremT]{Exercise}
% dark blue styles
\theoremstyle{darkbluenumbox}
% yellow styles
\theoremstyle{yellownumbox}
\newtheorem{remarkT}[theoremT]{Remark}
\theoremstyle{yellownumex}
\newtheorem{exampleT}[theoremT]{Example}
% red styles
\theoremstyle{rednumbox}
\newtheorem{definitionT}[theoremT]{Definition}
%\theoremstyle{rednumex}
%\newtheorem{remarkT}[theoremT]{Remark}
% black styles
\theoremstyle{blacknumex}
\newtheorem{problem}[theoremT]{Problem}

%----------------------------------------------------------------------------------------
%	DEFINITION OF COLORED BOXES
%----------------------------------------------------------------------------------------

\RequirePackage[framemethod=default]{mdframed} % Required for creating the theorem, definition, exercise and corollary boxes

% Theorem box
\newmdenv[skipabove=7pt,
skipbelow=7pt,
backgroundcolor=LightBlue!15,
linecolor=LightBlue,
rightline=false,
leftline=false,
topline=false,
bottomline=false,
innerleftmargin=5pt,
innerrightmargin=5pt,
innertopmargin=5pt,
leftmargin=0cm,
rightmargin=0cm,
innerbottommargin=5pt]{tBox}

% Lemma box
\newmdenv[skipabove=7pt,
skipbelow=7pt,
backgroundcolor=Yellow!15,
linecolor=Yellow,
rightline=false,
leftline=false,
topline=false,
bottomline=false,
innerleftmargin=5pt,
innerrightmargin=5pt,
innertopmargin=5pt,
leftmargin=0cm,
rightmargin=0cm,
innerbottommargin=5pt]{lBox}

% Exercise box
\newmdenv[skipabove=7pt,
skipbelow=7pt,
rightline=false,
leftline=true,
topline=false,
bottomline=false,
linecolor= tolMagenta,
innerleftmargin=5pt,
innerrightmargin=5pt,
innertopmargin=0pt,
innerbottommargin=0pt,
leftmargin=0cm,
rightmargin=0cm,
linewidth=2pt]{ecBox}

% Example box
\newmdenv[skipabove=7pt,
skipbelow=7pt,
rightline=false,
leftline=true,
topline=false,
bottomline=false,
linecolor=tolOrange,
innerleftmargin=5pt,
innerrightmargin=5pt,
innertopmargin=0pt,
innerbottommargin=0pt,
leftmargin=0cm,
rightmargin=0cm,
linewidth=2pt]{epBox}

% Definition box
\newmdenv[skipabove=7pt,
skipbelow=7pt,
rightline=false,
leftline=false,
topline=true,
bottomline=false,
backgroundcolor=tolBlue!15,
linecolor = tolBlue,
innerleftmargin=5pt,
innerrightmargin=5pt,
innertopmargin=5pt,
leftmargin=0cm,
rightmargin=0cm,
linewidth=4pt,
innerbottommargin=5pt]{dBox}

% Corollary box
\newmdenv[skipabove=7pt,
skipbelow=7pt,
rightline=false,
leftline=false,
topline=false,
bottomline=false,
linecolor=LightBlue,
backgroundcolor=LightBlue!15,
innerleftmargin=5pt,
innerrightmargin=5pt,
innertopmargin=5pt,
leftmargin=0cm,
rightmargin=0cm,
linewidth=4pt,
innerbottommargin=5pt]{cBox}

% Remark box
\newmdenv[skipabove=7pt,
skipbelow=7pt,
rightline=false,
leftline=false,
topline=true,
bottomline=false,
backgroundcolor=tolRed!15,
linecolor = tolRed,
innerleftmargin=5pt,
innerrightmargin=5pt,
innertopmargin=5pt,
leftmargin=0cm,
rightmargin=0cm,
linewidth=4pt,
innerbottommargin=5pt]{rBox}


\newenvironment{colourframe}[1][]{\begin{mdframed}[linewidth=2pt,linecolor=#1,backgroundcolor=black!2]}
  {\end{mdframed}}


% Creates an environment for each type of theorem and assigns it a theorem text style from the "Theorem Styles" section above and a colored box from above
\newenvironment{theorem}{\begin{tBox}\begin{theoremT}}{\end{theoremT}\end{tBox}}
\newenvironment{lemma}{\begin{lBox}\begin{lemmaT}}{\end{lemmaT}\end{lBox}}
\newenvironment{exercise}{\begin{ecBox}\begin{exerciseT}}{\end{exerciseT}\end{ecBox}}
\newenvironment{example}{\begin{epBox}\begin{exampleT}}{\end{exampleT}\end{epBox}}
\newenvironment{definition}{\begin{dBox}\begin{definitionT}}{\end{definitionT}\end{dBox}}
\newenvironment{corollary}{\begin{cBox}\begin{corollaryT}}{\end{corollaryT}\end{cBox}}
\newenvironment{remark}{\begin{rBox}\begin{remarkT}}{\end{remarkT}\end{rBox}}

%----------------------------------------------------------------------------------------
%	WARNING ENVIRONMENT -- currently not in use; previous REMARK environment
%----------------------------------------------------------------------------------------

%\newenvironment{warning}{\par\vspace{10pt}\small % Vertical white space above the remark and smaller font size
%\begin{list}{}{
%\leftmargin=35pt % Indentation on the left
%\rightmargin=25pt}\item\ignorespaces % Indentation on the right
%\makebox[-2.5pt]{\begin{tikzpicture}[overlay]
%\node[draw=Red!60,line width=1pt,circle,fill=Red!25,font=\sffamily\bfseries,inner sep=2pt,outer sep=0pt] at (-15pt,0pt){\textcolor{Red}{!!!}};\end{tikzpicture}} % Orange R in a circle
%\advance\baselineskip -1pt}{\end{list}\vskip5pt} % Tighter line spacing and white space after remark

%----------------------------------------------------------------------------------------
%	SECTION NUMBERING IN THE MARGIN
%----------------------------------------------------------------------------------------

\makeatletter
\renewcommand{\@seccntformat}[1]{\llap{\textcolor{DarkBlue}{\csname the#1\endcsname}\hspace{1em}}}
\renewcommand{\section}{\@startsection{section}{1}{\z@}
{-4ex \@plus -1ex \@minus -.4ex}
{1ex \@plus.2ex }
{\normalfont\large\sffamily\bfseries\textcolor{DarkBlue}}} % section font color
\renewcommand{\subsection}{\@startsection {subsection}{2}{\z@}
{-3ex \@plus -0.1ex \@minus -.4ex}
{0.5ex \@plus.2ex }
{\normalfont\sffamily\bfseries}}
\renewcommand{\subsubsection}{\@startsection {subsubsection}{3}{\z@}
{-2ex \@plus -0.1ex \@minus -.2ex}
{.2ex \@plus.2ex }
{\normalfont\small\sffamily\bfseries}}
\renewcommand\paragraph{\@startsection{paragraph}{4}{\z@}
{-2ex \@plus-.2ex \@minus .2ex}
{.1ex}
{\normalfont\small\sffamily\bfseries}}

%----------------------------------------------------------------------------------------
%	PART HEADINGS
%----------------------------------------------------------------------------------------

% Numbered part in the table of contents
\newcommand{\@mypartnumtocformat}[2]{%
	\setlength\fboxsep{0pt}%
	\noindent\colorbox{LightBlue!20}{\strut\parbox[c][.7cm]{\ecart}{\color{LightBlue!70}\Large\sffamily\bfseries\centering#1}}\hskip\esp\colorbox{LightBlue!40}{\strut\parbox[c][.7cm]{\linewidth-\ecart-\esp}{\Large\sffamily\centering#2}}%
}

% Unnumbered part in the table of contents
\newcommand{\@myparttocformat}[1]{%
	\setlength\fboxsep{0pt}%
	\noindent\colorbox{LightBlue!40}{\strut\parbox[c][.7cm]{\linewidth}{\Large\sffamily\centering#1}}%
}

\newlength\esp
\setlength\esp{4pt}
\newlength\ecart
\setlength\ecart{1.2cm-\esp}
\newcommand{\thepartimage}{}%
\newcommand{\partimage}[1]{\renewcommand{\thepartimage}{#1}}%
\def\@part[#1]#2{%
\ifnum \c@secnumdepth >-2\relax%
\refstepcounter{part}%
\addcontentsline{toc}{part}{\texorpdfstring{\protect\@mypartnumtocformat{\thepart}{#1}}{\partname~\thepart\ ---\ #1}}
\else%
\addcontentsline{toc}{part}{\texorpdfstring{\protect\@myparttocformat{#1}}{#1}}%
\fi%
\startcontents%
\markboth{}{}%
{\thispagestyle{empty}%
\begin{tikzpicture}[remember picture,overlay]%
\node at (current page.north west){\begin{tikzpicture}[remember picture,overlay]%
\fill[LightBlue!20](0cm,0cm) rectangle (\paperwidth,-\paperheight);
\node[anchor=north] at (4cm,-3.25cm){\color{LightBlue!40}\fontsize{220}{100}\sffamily\bfseries\thepart};
\node[anchor=south east] at (\paperwidth-1cm,-\paperheight+1cm){\parbox[t][][t]{8.5cm}{
\printcontents{l}{0}{\setcounter{tocdepth}{1}}% The depth to which the Part mini table of contents displays headings; 0 for chapters only, 1 for chapters and sections and 2 for chapters, sections and subsections
}};
\node[anchor=north east] at (\paperwidth-1.5cm,-3.25cm){\parbox[t][][t]{15cm}{\strut\raggedleft\color{white}\fontsize{30}{30}\sffamily\bfseries#2}};
\end{tikzpicture}};
\end{tikzpicture}}%
\@endpart}
\def\@spart#1{%
\startcontents%
\phantomsection
{\thispagestyle{empty}%
\begin{tikzpicture}[remember picture,overlay]%
\node at (current page.north west){\begin{tikzpicture}[remember picture,overlay]%
\fill[LightBlue!20](0cm,0cm) rectangle (\paperwidth,-\paperheight);
\node[anchor=north east] at (\paperwidth-1.5cm,-3.25cm){\parbox[t][][t]{15cm}{\strut\raggedleft\color{white}\fontsize{30}{30}\sffamily\bfseries#1}};
\end{tikzpicture}};
\end{tikzpicture}}
\addcontentsline{toc}{part}{\texorpdfstring{%
\setlength\fboxsep{0pt}%
\noindent\protect\colorbox{LightBlue!40}{\strut\protect\parbox[c][.7cm]{\linewidth}{\Large\sffamily\protect\centering #1\quad\mbox{}}}}{#1}}%
\@endpart}
\def\@endpart{\vfil\newpage
\if@twoside
\if@openright
\null
\thispagestyle{empty}%
\newpage
\fi
\fi
\if@tempswa
\twocolumn
\fi}

%----------------------------------------------------------------------------------------
%	CHAPTER HEADINGS
%----------------------------------------------------------------------------------------

% A switch to conditionally include a picture, implemented by Christian Hupfer
\newif\ifusechapterimage
\usechapterimagetrue
\newcommand{\thechapterimage}{}%
\newcommand{\chapterimage}[1]{\ifusechapterimage\renewcommand{\thechapterimage}{#1}\fi}%
\newcommand{\autodot}{.}
\def\@makechapterhead#1{%
{\parindent \z@ \raggedright \normalfont
\ifnum \c@secnumdepth >\m@ne
\if@mainmatter
\begin{tikzpicture}[remember picture,overlay]
\node at (current page.north west)
{\begin{tikzpicture}[remember picture,overlay]
\node[anchor=north west,inner sep=0pt] at (0,0) {\ifusechapterimage\includegraphics[width=\paperwidth]{\thechapterimage}\fi};
\draw[anchor=west] (\Gm@lmargin,-9cm) node [line width=2pt,fill=LightBlue!20!white,fill opacity=0.5,inner sep=15pt]{\strut\makebox[22cm]{}};
\draw[anchor=west] (\Gm@lmargin+.3cm,-9cm) node {\Huge\sffamily\bfseries\color{LightBlue}\thechapter\autodot~#1\strut};
\end{tikzpicture}};
\end{tikzpicture}
\else
\begin{tikzpicture}[remember picture,overlay]
\node at (current page.north west)
{\begin{tikzpicture}[remember picture,overlay]
\node[anchor=north west,inner sep=0pt] at (0,0) {\ifusechapterimage\includegraphics[width=\paperwidth]{\thechapterimage}\fi};
\draw[anchor=west] (\Gm@lmargin,-9cm) node [line width=2pt,fill=LightBlue!20!white,fill opacity=0.5,inner sep=15pt]{\strut\makebox[22cm]{}};
\draw[anchor=west] (\Gm@lmargin+.3cm,-9cm) node {\Huge\sffamily\bfseries\color{LightBlue}#1\strut};
\end{tikzpicture}};
\end{tikzpicture}
\fi\fi\par\vspace*{270\p@}}}

%-------------------------------------------

\def\@makeschapterhead#1{%
\begin{tikzpicture}[remember picture,overlay]
\node at (current page.north west)
{\begin{tikzpicture}[remember picture,overlay]
\node[anchor=north west,inner sep=0pt] at (0,0) {\ifusechapterimage\includegraphics[width=\paperwidth]{\thechapterimage}\fi};
\draw[anchor=west] (\Gm@lmargin,-9cm) node [line width=2pt,fill=LightBlue!20!white,fill opacity=0.5,inner sep=15pt]{\strut\makebox[22cm]{}};
\draw[anchor=west] (\Gm@lmargin+.3cm,-9cm) node {\huge\sffamily\bfseries\color{LightBlue}#1\strut};
\end{tikzpicture}};
\end{tikzpicture}
\par\vspace*{270\p@}}
\makeatother

%----------------------------------------------------------------------------------------
%	SECTION HEADINGS
%----------------------------------------------------------------------------------------

\usepackage{titlesec}

\titleformat{\section}
  {\sffamily\Large\bfseries\color{DarkBlue}}{\thesection.}{1em}{}
  
  


%----------------------------------------------------------------------------------------
%	SUBFILES
%----------------------------------------------------------------------------------------

\usepackage[mode=buildnew]{standalone} % Required for including TiKz pictures with pre-compiling; load this package as late as possible

%----------------------------------------------------------------------------------------
%	CUSTOM MATH COMMANDS
%----------------------------------------------------------------------------------------

\newcommand{\declarecommand}[1]{\providecommand{#1}{}\renewcommand{#1}}

% Sets
\declarecommand{\R}{\mathbb{R}}
\declarecommand{\Q}{\mathbb{Q}}
\declarecommand{\Z}{\mathbb{Z}}
\declarecommand{\N}{\mathbb{N}}
\declarecommand{\C}{\mathbb{C}}
\declarecommand{\emptyset}{\varnothing}

% Operators
\DeclareMathOperator{\Int}{int}
\DeclareMathOperator{\Cl}{cl}
\DeclareMathOperator{\Span}{span} % lower case 'span' conflicts with tabular environment
\DeclareMathOperator{\Img}{img}
\DeclareMathOperator{\Ker}{ker}
\DeclareMathOperator{\Id}{id}
\DeclareMathOperator{\Rref}{rref}
\DeclareMathOperator{\Rank}{rank}
\DeclareMathOperator{\Null}{null}
\DeclareMathOperator{\Nullity}{nullity}
\DeclareMathOperator{\Proj}{proj}
\DeclareMathOperator{\Curl}{curl}
\DeclareMathOperator{\Div}{div}
\DeclareMathOperator{\Dim}{dim}
\DeclareMathOperator{\Grad}{grad}
\DeclareMathOperator{\Area}{area}
\DeclareMathOperator{\Vol}{vol}
\DeclareMathOperator{\Length}{length}
\DeclarePairedDelimiter\abs{\lvert}{\rvert}
\DeclarePairedDelimiter\norm{\lVert}{\rVert}

% Matrices
\newcommand{\mat}[1]{\begin{bmatrix*}[r]#1\end{bmatrix*}}
\newcommand{\matc}[1]{\begin{bmatrix}#1\end{bmatrix}}

% Other
\declarecommand{\epsilon}{\varepsilon}
\declarecommand{\ds}{\displaystyle}



%----------------------------------------------------------------------------------------
%	DIAGRAMS
%----------------------------------------------------------------------------------------

% integration graphic
\newcommand{\puntos}[2]{
	\draw[thick] (-5,0) to (5,0);
	
	\draw[decorate, decoration={brace, amplitude=10pt}, yshift=10pt] (#1-3.5,-0) -- (#1-0,-0) 
		node [Red,midway,yshift=20pt]{lower sums};
	
	\draw[Red, fill] (#1-2.5,0) circle [radius=0.08] ;
	\draw[Red, fill] (#1-3.3,0) circle [radius=0.08] ;
	\draw[Red, fill] (#1-2.0,0) circle [radius=0.08] ;
	\draw[Red, fill] (#1-1.8,0) circle [radius=0.08] ;
	\draw[Red, fill] (#1-1.3,0) circle [radius=0.08] ;
	\draw[Red, fill] (#1-0.8,0) circle [radius=0.08] ;
	\draw[Red, fill] (#1-0.5,0) circle [radius=0.08] ;
	\draw[Red, fill] (#1-0.2,0) circle [radius=0.08] ; s
	\draw[Red, fill] (#1-0.05,0) circle [radius=0.08] ;

	\draw[decorate, decoration={brace, amplitude=10pt}, yshift=10pt] (#2+0,0) -- (#2+3.5,0) 
		node [LightBlue,midway,yshift=20pt]{upper sums};
		
	\draw[LightBlue, fill] (#2+2.7,0) circle [radius=0.08] ;
	\draw[LightBlue, fill] (#2+3.1,0) circle [radius=0.08] ;
	\draw[LightBlue, fill] (#2+1.8,0) circle [radius=0.08] ;
	\draw[LightBlue, fill] (#2+1.5,0) circle [radius=0.08] ;
	\draw[LightBlue, fill] (#2+1.2,0) circle [radius=0.08] ;
	\draw[LightBlue, fill] (#2+0.7,0) circle [radius=0.08] ;
	\draw[LightBlue, fill] (#2+0.4,0) circle [radius=0.08] ;
	\draw[LightBlue, fill] (#2+0.2,0) circle [radius=0.08] ;
	\draw[LightBlue, fill] (#2+0.03,0) circle [radius=0.08] ;
}

% integration
\newcommand{\puntosmas}[2]{
	\puntos{#1}{#2}
	\draw[very thick, ->] (#1-3.5,-1)--(#1-0,-1) node[black,midway,yshift=-20pt]{finer partitions};
	\draw[very thick, ->] (#2+3.5,-1)--(#2+0,-1) node[black,midway,yshift=-20pt]{finer partitions};
}



